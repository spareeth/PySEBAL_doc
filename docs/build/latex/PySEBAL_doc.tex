%% Generated by Sphinx.
\def\sphinxdocclass{report}
\documentclass[letterpaper,10pt,english]{sphinxmanual}
\ifdefined\pdfpxdimen
   \let\sphinxpxdimen\pdfpxdimen\else\newdimen\sphinxpxdimen
\fi \sphinxpxdimen=.75bp\relax

\PassOptionsToPackage{warn}{textcomp}
\usepackage[utf8]{inputenc}
\ifdefined\DeclareUnicodeCharacter
% support both utf8 and utf8x syntaxes
  \ifdefined\DeclareUnicodeCharacterAsOptional
    \def\sphinxDUC#1{\DeclareUnicodeCharacter{"#1}}
  \else
    \let\sphinxDUC\DeclareUnicodeCharacter
  \fi
  \sphinxDUC{00A0}{\nobreakspace}
  \sphinxDUC{2500}{\sphinxunichar{2500}}
  \sphinxDUC{2502}{\sphinxunichar{2502}}
  \sphinxDUC{2514}{\sphinxunichar{2514}}
  \sphinxDUC{251C}{\sphinxunichar{251C}}
  \sphinxDUC{2572}{\textbackslash}
\fi
\usepackage{cmap}
\usepackage[T1]{fontenc}
\usepackage{amsmath,amssymb,amstext}
\usepackage{babel}



\usepackage{times}
\expandafter\ifx\csname T@LGR\endcsname\relax
\else
% LGR was declared as font encoding
  \substitutefont{LGR}{\rmdefault}{cmr}
  \substitutefont{LGR}{\sfdefault}{cmss}
  \substitutefont{LGR}{\ttdefault}{cmtt}
\fi
\expandafter\ifx\csname T@X2\endcsname\relax
  \expandafter\ifx\csname T@T2A\endcsname\relax
  \else
  % T2A was declared as font encoding
    \substitutefont{T2A}{\rmdefault}{cmr}
    \substitutefont{T2A}{\sfdefault}{cmss}
    \substitutefont{T2A}{\ttdefault}{cmtt}
  \fi
\else
% X2 was declared as font encoding
  \substitutefont{X2}{\rmdefault}{cmr}
  \substitutefont{X2}{\sfdefault}{cmss}
  \substitutefont{X2}{\ttdefault}{cmtt}
\fi


\usepackage[Bjarne]{fncychap}
\usepackage{sphinx}

\fvset{fontsize=\small}
\usepackage{geometry}

% Include hyperref last.
\usepackage{hyperref}
% Fix anchor placement for figures with captions.
\usepackage{hypcap}% it must be loaded after hyperref.
% Set up styles of URL: it should be placed after hyperref.
\urlstyle{same}
\addto\captionsenglish{\renewcommand{\contentsname}{Contents:}}

\usepackage{sphinxmessages}
\setcounter{tocdepth}{2}



\title{PySEBAL\_doc Documentation}
\date{Dec 06, 2019}
\release{0.1}
\author{Sajid Pareeth}
\newcommand{\sphinxlogo}{\vbox{}}
\renewcommand{\releasename}{Release}
\makeindex
\begin{document}

\pagestyle{empty}
\sphinxmaketitle
\pagestyle{plain}
\sphinxtableofcontents
\pagestyle{normal}
\phantomsection\label{\detokenize{index::doc}}



\chapter{PySEBAL Installation}
\label{\detokenize{installation:pysebal-installation}}\label{\detokenize{installation::doc}}
PySEBAL is a python library to compute Actual EvapoTranspiration (ETa) and other related variables using SEBAL model. Following specifications are recommended to run PySEBAL.


\section{Computing requirements}
\label{\detokenize{installation:computing-requirements}}

\subsection{Hardware}
\label{\detokenize{installation:hardware}}\begin{itemize}
\item {} 
CPU with 2 cores and \textgreater{} 2GHz processor

\item {} 
Minimum of 8 GB RAM

\item {} 
Storage space of 10 - 20 GB if processing multiple landsat tiles

\end{itemize}


\subsection{Operating systems}
\label{\detokenize{installation:operating-systems}}\begin{itemize}
\item {} 
Windows 7/10 (Windows 8, 8.1 should also work, provided dependencies are met)

\item {} 
Linux (Tested in Ubuntu 18.04 LTS, other Linux OS should also work)

\end{itemize}

PySEBAL can run using both Python 2 and 3, (tested in \textgreater{} 2.7 and \textgreater{} 3.6).


\section{Source code}
\label{\detokenize{installation:source-code}}
The PySEBAL library is hosted in a publicly available github repository. The library can be downloaded from \sphinxhref{https://github.com/spareeth/PySEBAL\_dev}{here}

\begin{figure}[htbp]
\centering

\noindent\sphinxincludegraphics[width=400\sphinxpxdimen]{{git1}.png}
\end{figure}

\begin{DUlineblock}{0em}
\item[] Once it is downloaded, unzip it (use \sphinxstyleemphasis{extract here}) into your \sphinxcode{\sphinxupquote{D:\textbackslash{}}} drive or any drive other than \sphinxcode{\sphinxupquote{C:\textbackslash{}}} drive.
\end{DUlineblock}

\begin{DUlineblock}{0em}
\item[] The directory structure after download and unzip should like like below.
\end{DUlineblock}

\begin{figure}[htbp]
\centering

\noindent\sphinxincludegraphics[width=400\sphinxpxdimen]{{git2}.png}
\end{figure}


\section{Installation in Windows}
\label{\detokenize{installation:installation-in-windows}}
The PySEBAL library has multiple dependencies to support spatial data processing and computing. All the required libraries are open source. We reccomend using the OSGeo4W installer and environment to install all the dependencies and to run PySEBAL library in command line.


\subsection{Installing dependencies}
\label{\detokenize{installation:installing-dependencies}}
\begin{DUlineblock}{0em}
\item[] Following steps explain the installation procedure:
\item[] \sphinxstylestrong{Step1 - Download the OSGeo4W installer}
\end{DUlineblock}

\begin{DUlineblock}{0em}
\item[] Get the OSGeo4W installer (\sphinxurl{https://trac.osgeo.org/osgeo4w/}
from this \sphinxhref{http://download.osgeo.org/osgeo4w/osgeo4w-setup-x86\_64.exe}{link} (64 bit)
Or get it from the usb stick provided.
\end{DUlineblock}

\begin{DUlineblock}{0em}
\item[] \sphinxstylestrong{Step2 - Install the dependencies}
\end{DUlineblock}

\begin{DUlineblock}{0em}
\item[] Double click the OSGeo4W installer
\end{DUlineblock}
\begin{itemize}
\item {} 
Select “advanced install” and click “Next”

\end{itemize}

\begin{figure}[htbp]
\centering

\noindent\sphinxincludegraphics[width=400\sphinxpxdimen]{{osgeo1}.png}
\end{figure}
\begin{itemize}
\item {} 
In this step there are two options, choose option 1 or 2.
\begin{quote}
\begin{enumerate}
\sphinxsetlistlabels{\arabic}{enumi}{enumii}{}{.}%
\item {} 
Select “Install from Local Directory” and click “Next”, if you want to install from the source libraries provided to you in USB.

\end{enumerate}
\begin{quote}

\begin{figure}[htbp]
\centering

\noindent\sphinxincludegraphics[width=400\sphinxpxdimen]{{osgeo2b}.png}
\end{figure}
\end{quote}
\begin{enumerate}
\sphinxsetlistlabels{\arabic}{enumi}{enumii}{}{.}%
\setcounter{enumi}{1}
\item {} 
Select “Install from internet” and click “Next”, you must be connected to a good internet.

\end{enumerate}
\begin{quote}

\begin{figure}[htbp]
\centering

\noindent\sphinxincludegraphics[width=400\sphinxpxdimen]{{osgeo2a}.png}
\end{figure}
\end{quote}
\end{quote}

\item {} 
In this step select the root directory and access to users, keep default settings, and optionally “Create icon on Desktop” for easy access.

\end{itemize}

\begin{figure}[htbp]
\centering

\noindent\sphinxincludegraphics[width=400\sphinxpxdimen]{{osgeo3}.png}
\end{figure}
\begin{itemize}
\item {} 
Here choose the folder with local repository (provided to you in USB) if you have selected option 1 in the previous step \sphinxstylestrong{or} choose the folder (default option) to download the libraries from internet if you have selected option 2 in the previous step and click “Next”.

\end{itemize}

\begin{figure}[htbp]
\centering

\noindent\sphinxincludegraphics[width=400\sphinxpxdimen]{{osgeo4}.png}
\end{figure}
\begin{itemize}
\item {} 
In case of option 2 “Install from internet” in the previous step, select the default option “Direct Connection” and click “Next”.

\item {} 
In case of option 2 “Install from internet” in the previous step, select the default option “\sphinxurl{http://download.osgeo.org}” as the download site and click “Next”.

\end{itemize}
\begin{itemize}
\item {} 
In this step, search for the following packages \sphinxstylestrong{one by one}, and select the appropriate (latest) versions by clicking the \sphinxincludegraphics{{osgeo_icon}.png} icon under the column \sphinxstylestrong{New}. Check under the \sphinxstylestrong{Package} column if you are selecting exact library as stated below.

\end{itemize}

\begin{sphinxadmonition}{warning}{Warning:}
Do not click next before selecting all the packages listed below !
\end{sphinxadmonition}

\begin{DUlineblock}{0em}
\item[] The required libraries are:
\end{DUlineblock}
\begin{quote}
\begin{itemize}
\item {} 
qgis-ltr

\item {} 
grass

\item {} 
qgis-ltr-grass-plugin7

\item {} 
msys

\item {} 
pyproj (select both the packages for python 2 \& 3)

\item {} 
pandas (select all four packages python 2 \& 3 , pandas and geopandas packages)

\item {} 
scipy (select both the packages for python 2 \& 3)

\item {} 
tcltk (select both the packages for python 2 \& 3)

\end{itemize}

\begin{DUlineblock}{0em}
\item[] Click “Next” and finish the installation
\end{DUlineblock}
\end{quote}

\begin{DUlineblock}{0em}
\item[] \sphinxstylestrong{Step3 - Install additional dependencies}
\end{DUlineblock}
\begin{itemize}
\item {} 
In the program menu search for “OSGeo4W Shell” or if you have selected “Create icon on Desktop” option in the previous step, it should be in the desktop. Now open “OSGeo4W Shell”

\end{itemize}

\begin{figure}[htbp]
\centering

\noindent\sphinxincludegraphics[width=200\sphinxpxdimen]{{shell1}.png}
\end{figure}
\begin{itemize}
\item {} 
In the OSGeo4W Shell type in the following commands to install packages - \sphinxstyleemphasis{setuptools, openpyxl, netCDF4, joblib}

\end{itemize}

\begin{sphinxVerbatim}[commandchars=\\\{\},numbers=left,firstnumber=1,stepnumber=1]
pip install setuptools
pip install openpyxl netCDF4 joblib
\end{sphinxVerbatim}


\subsection{Setting environment variables}
\label{\detokenize{installation:setting-environment-variables}}
\begin{DUlineblock}{0em}
\item[] \sphinxstylestrong{Steps}
\end{DUlineblock}
\begin{itemize}
\item {} 
Right click “This PC” in Windows 10 \sphinxstylestrong{OR} “My Computer” in windows 7, go to \sphinxstyleemphasis{Properties} -\textgreater{} \sphinxstyleemphasis{Advaced system settings} -\textgreater{} \sphinxstyleemphasis{Advanced} tab -\textgreater{} \sphinxstyleemphasis{Environment variables} -\textgreater{} \sphinxstyleemphasis{System variables}.

\item {} 
Click new and add four new system variables. Add the \sphinxstylestrong{Variable name} and \sphinxstylestrong{Variable value} as shown below.

\end{itemize}
\begin{itemize}
\item {} 
\sphinxstylestrong{GDAL\_DATA} set to \sphinxcode{\sphinxupquote{C:\textbackslash{}OSGeo4W64\textbackslash{}share\textbackslash{}epsg\_csv}}

\item {} 
\sphinxstylestrong{PYTHONPATH} set to \sphinxcode{\sphinxupquote{C:\textbackslash{}OSGeo4W64\textbackslash{}apps\textbackslash{}Python27\textbackslash{}Scripts}}

\item {} 
\sphinxstylestrong{PYTHONHOME} set to \sphinxcode{\sphinxupquote{C:\textbackslash{}OSGeo4W64\textbackslash{}apps\textbackslash{}Python27}}

\item {} 
\sphinxstylestrong{SEBAL} set to \sphinxcode{\sphinxupquote{C:\textbackslash{}OSGeo4W64\textbackslash{}bin}}

\end{itemize}
\begin{itemize}
\item {} 
Edit the variable \sphinxstylestrong{Path} in the \sphinxstylestrong{System variables} to add the path \sphinxcode{\sphinxupquote{C:\textbackslash{}OSGeo4W64\textbackslash{}bin}} to the end followed by a semicolon (;) in windows 7 \sphinxstylestrong{OR} add this path as a new line in the path variable in Windows 10.

\end{itemize}


\subsection{Test installation}
\label{\detokenize{installation:test-installation}}
To test whether the PySEBAL will run, open OSGeo4W Shell, and type following commands.

\begin{sphinxVerbatim}[commandchars=\\\{\},numbers=left,firstnumber=1,stepnumber=1]
\PYG{c+c1}{\PYGZsh{} After each command click enter}
\PYG{c+c1}{\PYGZsh{} Any line starting with \PYGZsq{}\PYGZsh{}\PYGZsq{} is comment line}
\PYG{c+c1}{\PYGZsh{} Change drive}
\PYG{n}{D}\PYG{p}{:}
\PYG{n}{cd} \PYG{n}{PySEBAL\PYGZus{}dev}\PYGZbs{}\PYG{n}{SEBAL}
\PYG{c+c1}{\PYGZsh{} open python}
\PYG{n}{python}
\PYG{c+c1}{\PYGZsh{} import one of the PySEBAL Script}
\PYG{k+kn}{import} \PYG{n+nn}{pysebal\PYGZus{}py2}
\PYG{c+c1}{\PYGZsh{} If there are no errors, the installation is successful}
\PYG{c+c1}{\PYGZsh{} To exit from python}
\PYG{n+nb}{exit}\PYG{p}{(}\PYG{p}{)}
\end{sphinxVerbatim}


\section{Installation in Linux}
\label{\detokenize{installation:installation-in-linux}}
The below steps are tested in Ubuntu 18.04 LTS, it should also work in other Linux distibutions, you may have to adapt some of the installation steps accordingly. This is also valid for installation in \sphinxstylestrong{Bash for Windows} app with Ubuntu inside windows 10.

\begin{sphinxadmonition}{note}{Note:}
You can check the python version using the command \sphinxcode{\sphinxupquote{python -{-}version}} in a terminal
\end{sphinxadmonition}


\subsection{Installing dependencies}
\label{\detokenize{installation:id1}}
The dependencies packages are same as those in windows except for msys. We also install git to download and clone the PySEBAL\_dev repository.

Open a Terminal and type in following commands to install required packages. You should have admin rights to install packages.

\begin{sphinxadmonition}{warning}{Warning:}
Please remove all the QGIS and GRASS packages you may have installed from other repositories before doing the update.
\end{sphinxadmonition}

\begin{sphinxVerbatim}[commandchars=\\\{\},numbers=left,firstnumber=1,stepnumber=1]
\PYG{c+c1}{\PYGZsh{} After each command click enter}
\PYG{c+c1}{\PYGZsh{} Any line starting with \PYGZsq{}\PYGZsh{}\PYGZsq{} is comment line}
\PYG{c+c1}{\PYGZsh{} Install git}
sudo apt\PYGZhy{}get install git
\PYG{c+c1}{\PYGZsh{} Add a PPA to install required GIS softwares}
sudo add\PYGZhy{}apt\PYGZhy{}repository ppa:ubuntugis/ubuntugis\PYGZhy{}unstable
sudo apt\PYGZhy{}get update
\PYG{c+c1}{\PYGZsh{} Install qgis and qgis\PYGZhy{}grass plugin}
sudo apt\PYGZhy{}get install qgis qgis\PYGZhy{}plugin\PYGZhy{}grass
\PYG{c+c1}{\PYGZsh{} Install GRASS GIS and required packages}
sudo add\PYGZhy{}apt\PYGZhy{}repository ppa:ubuntugis/ppa
sudo add\PYGZhy{}apt\PYGZhy{}repository ppa:grass/grass\PYGZhy{}stable
sudo apt\PYGZhy{}get update
sudo apt\PYGZhy{}get install grass78
\PYG{c+c1}{\PYGZsh{} Install openpyxl, netCDF4, joblib packages}
\PYG{c+c1}{\PYGZsh{} For python 3, use pip3 to install ....}
pip install openpyxl netCDF4 joblib
\end{sphinxVerbatim}

For other Linux distributions there is detailed instruction to install qgis \sphinxhref{https://qgis.org/en/site/forusers/alldownloads.html}{here} and grass gis \sphinxhref{https://grass.osgeo.org/download/software/linux/}{here}.


\subsection{Download source code}
\label{\detokenize{installation:download-source-code}}
Open a terminal and type in following git command to download the PySEBAL\_dev repository.

\begin{sphinxVerbatim}[commandchars=\\\{\},numbers=left,firstnumber=1,stepnumber=1]
\PYG{c+c1}{\PYGZsh{} After each command click enter}
\PYG{c+c1}{\PYGZsh{} Any line starting with \PYGZsq{}\PYGZsh{}\PYGZsq{} is comment line}
\PYG{c+c1}{\PYGZsh{} change to working directory,}
\PYG{c+c1}{\PYGZsh{} /mnt/d if you are accessing windows D: drive from linux. For example \PYGZdq{}bash for windows\PYGZdq{} in windows 10}
\PYG{n+nb}{cd} /mnt/d
\PYG{c+c1}{\PYGZsh{} Clone the PySEBAL\PYGZus{}dev repository}
git clone https://github.com/spareeth/PySEBAL\PYGZus{}dev.git
\end{sphinxVerbatim}


\subsection{Testing installation}
\label{\detokenize{installation:testing-installation}}
Open a terminal and type in following codes to test if the installation is successful.

\begin{sphinxVerbatim}[commandchars=\\\{\},numbers=left,firstnumber=1,stepnumber=1]
\PYG{c+c1}{\PYGZsh{} After each command click enter}
\PYG{c+c1}{\PYGZsh{} Any line starting with \PYGZsq{}\PYGZsh{}\PYGZsq{} is comment line}
\PYG{c+c1}{\PYGZsh{} change to the PySEBAL\PYGZus{}dev directory, assuming that the repository is cloned in /mnt/d}
\PYG{n+nb}{cd} /mnt/d/PySEBAL\PYGZus{}dev/SEBAL
\PYG{c+c1}{\PYGZsh{} List the files inside this folder}
ls
\PYG{c+c1}{\PYGZsh{} Open Python}
python
import pysebal\PYGZus{}py2
\PYG{c+c1}{\PYGZsh{} If there are no errors, the installation is successful}
\PYG{c+c1}{\PYGZsh{} To exit from python (ctrl\PYGZhy{}d)}
exit\PYG{o}{(}\PYG{o}{)}
\end{sphinxVerbatim}


\section{Test run PySEBAL}
\label{\detokenize{installation:test-run-pysebal}}
Once PySEBAL is installed, we can run the PySEBAL code using the test data provided with the PySEBAL\_dev library. The test data is located in the folder \sphinxcode{\sphinxupquote{PySEBAL\_dev\textbackslash{}test\_data}}. If you have installed PySEBAL in \sphinxcode{\sphinxupquote{D:}} drive then it should be \sphinxcode{\sphinxupquote{D:\textbackslash{}PySEBAL\_dev\textbackslash{}test\_data}}.

\begin{DUlineblock}{0em}
\item[] Assuming that PySEBAL\_dev is in \sphinxcode{\sphinxupquote{D:}} drive, Let us run the library with test data.
\end{DUlineblock}

\begin{DUlineblock}{0em}
\item[] Open a OSGeo4W Shell and change the directory to \sphinxcode{\sphinxupquote{D:\textbackslash{}PySEBAL\_dev\textbackslash{}SEBAL}} and follow the commands given below.
\end{DUlineblock}

\begin{DUlineblock}{0em}
\item[] \sphinxstylestrong{In Windows}
\end{DUlineblock}

\begin{sphinxVerbatim}[commandchars=\\\{\},numbers=left,firstnumber=1,stepnumber=1]
\PYG{c+c1}{\PYGZsh{} After each command click enter}
\PYG{c+c1}{\PYGZsh{} Any line starting with \PYGZsq{}\PYGZsh{}\PYGZsq{} is comment line}
\PYG{c+c1}{\PYGZsh{} change to the PySEBAL\PYGZus{}dev\PYGZbs{}SEBAL directory}
\PYG{n+nb}{cd} D:\PYG{l+s+se}{\PYGZbs{}P}ySEBAL\PYGZus{}dev\PYG{l+s+se}{\PYGZbs{}S}EBAL
\PYG{c+c1}{\PYGZsh{} Run the PySEBAL script}
python Run\PYGZus{}py2.py
\end{sphinxVerbatim}

\begin{DUlineblock}{0em}
\item[] \sphinxstylestrong{In Linux}
\end{DUlineblock}

\begin{sphinxVerbatim}[commandchars=\\\{\},numbers=left,firstnumber=1,stepnumber=1]
\PYG{c+c1}{\PYGZsh{} After each command click enter}
\PYG{c+c1}{\PYGZsh{} Any line starting with \PYGZsq{}\PYGZsh{}\PYGZsq{} is comment line}
\PYG{c+c1}{\PYGZsh{} change to the PySEBAL\PYGZus{}dev\PYGZbs{}SEBAL directory}
\PYG{n+nb}{cd} \PYG{l+s+se}{\PYGZbs{}m}nt\PYG{l+s+se}{\PYGZbs{}d}\PYG{l+s+se}{\PYGZbs{}P}ySEBAL\PYGZus{}dev\PYG{l+s+se}{\PYGZbs{}S}EBAL
\PYG{c+c1}{\PYGZsh{} Run the PySEBAL script}
python Run\PYGZus{}py2.py
\end{sphinxVerbatim}

After the above commands, there will be a \sphinxcode{\sphinxupquote{output}} folder inside \sphinxcode{\sphinxupquote{D:\textbackslash{}PySEBAL\textbackslash{}test\_data}} with the following structure.

\begin{figure}[htbp]
\centering

\noindent\sphinxincludegraphics[width=200\sphinxpxdimen]{{pysebal_folderstr}.png}
\end{figure}

\begin{sphinxadmonition}{warning}{Warning:}
If PySEBAL\_dev is not in \sphinxcode{\sphinxupquote{D:}} drive, adapt changes to the path in above commands accordingly.
\end{sphinxadmonition}

\begin{sphinxadmonition}{note}{Note:}
Now go to the folder \sphinxcode{\sphinxupquote{D:\textbackslash{}PySEBAL\_dev\textbackslash{}test\_data\textbackslash{}output\textbackslash{}Output\_evapotranspiration}} and check the daily ETa map (\sphinxstyleemphasis{L8\_ETact\_24\_30m\_2014\_03\_10\_069.tif}) in QGIS.
\end{sphinxadmonition}



\renewcommand{\indexname}{Index}
\printindex
\end{document}